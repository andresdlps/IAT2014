The action description defines all the necessary information that are required to generate emotive actions and personalize the system. This system is divided into parts: general, which described the information to generate the emotive actions. And Theatre, which describes the script and the current character that should be portrayed by 
the robot.
\subsection{Theatre}
The theatre configuration is the component responsible for store and manage the traditional theatre artefact like the script, the character descriptor, the scenario and even the directors guidelines.
\subsection{General}
The general configuration is components that are not limited just a theatre, and there are divided in three groups:
\begin{itemize}
	\item \textit{Actions}: describe the information about the actions that are available in the system. This information enable the user to create new actions based on the existing ones. 
	\item \textit{Platform}: specify which actions could perform each platform, which allow the use the same script in different platforms.
	\item \textit{Emotions}: have the information in how modify the actions to convey emotions. 
\end{itemize} 
