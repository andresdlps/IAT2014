Literature about robot actors is not wide. Although there are works that have used robots in theatrical representations, these robots are often used as props. Other works have focused on the development of systems that could interact with the audience \cite{Mavridis2009,Breazeal2003}. The most advanced is the standup comedy implemented by Knight \cite{Knight2011b,Knight2011}. In this, a NAO platform~\cite{NAO2013} sits in front of a whole auditorium. The robot tells a joke and selects autonomously the next joke based on the reaction of the audience. To get this information, the robot relies on cameras, colored paddles and a microphone. The robot performs basic actions to add some expressiveness to the joke, but it does not project any emotion.\\
Other works~\cite{Wurst2002,Bruce2002} have adopted theatre as a place to implement simple actors with all information (even objects' position) hard-coded, and no interaction with people. Although these robots could be perceived as autonomous, they do not make any adjustment during their performance, losing realism, e.g., if the other robot does not play in its precise place. \\
Complex performances have been given in 2011, when \textit{Roboscopie}~\cite{Roboscopie2012,Lemaignan2012} was presented. Roboscopie story involved one person and one robot. The robot could navigate autonomously in its environment and build a 3D model of it. But, the human-robot coordination was done manually during the presentation, and the position of some objects was already known by the robot.\\
Fahn and collaborators worked with humanoid robots as actors~\cite{Fan2009,Fan2013}. They believe humanoid robots are suitable as theatre actors more than other kind of robots. They have developed two humanoid robots, Janet and Thomas~\cite{Fan2009}, and two wheeled robots, Pica and Ringo \cite{Hsu2007}. These robots are capable to perform autonomously many actions such as: draw people, jazz drumming, marionette operating, and notation reading and singing. However, their humanoid robots have problems with the amount of computers needed to control them. The high complexity to control the humanoid platforms and the lack of expression in their action, except for their faces, make this approach unsuitable to study emotion generation.\\
Trying to add some theatrical realism to robots, Breazeal and collaborators~\cite{Breazeal2008} designed and implemented a system to control a lamp. The main characteristic of this lamp is that it could be controlled by just one person, which could select the focus point where the lamp must look at. This little thing improved the credibility that the lamp was listening to the person that was speaking to it.\\
With the idea to familiarize people with robots, \textit{Robots actors project} was created~\cite{Zaven2012,Torres2013}, where Wakamaru~\cite{Wakamaru2013} and Geminoid F~\cite{Ishiguro2013} robots were used in the play. Unfortunately, deeper information about this implementation is not available, although from videos is possible to infer that the play has been designed for robots and robots seem remotely driven.