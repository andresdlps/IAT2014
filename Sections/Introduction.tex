Theatre has been suggested as a suitable environment to test timing and expressiveness in robotics~\cite{Breazeal2003,Fan2013,Hoffman2009,lu2011position,Pinhanez97}. However, few works have been presented using theatre as environment to build and test sociability in robots. Some have used theatre to study timing in human-human interaction~\cite{Knight2011,Knight2011b}, others have focused on humanoid platforms that could be used as actors~\cite{Fan2009,Fan2013}, but none of them have worked on how to use theatre to improve emotion expression in totally autonomous robots.\\ 

The introduction section show the mainly issues that come with the praxis of drama and escenic arts. This section has two part:

The first one, describes the entire world of human-robot interaction and select which features this architecture aims to achieve.

The second part describes this works in terms of the described features like cooperative multi-agent robotics, emotional expresiveness and theatrical representation.

This papers is organized as follows. In section~\ref{sec:relatedwork} is reviewed the use of robots in theatre and section~\ref{sec:theatre} gives the highlights to be selected theatre as a test bed. Section~\ref{sec:architecture} describes the generalities of the architecture, and sections\ref{sec:description},~\ref{sec:belief},~\ref{sec:action}, and~\ref{sec:motivation} gives a more detail about each sub-system that conforms the architecture. Conclusions and further work are presented in section~\ref{sec:conclusions}. 

