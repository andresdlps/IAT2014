Theatre has been suggested as a suitable environment to test timing and expressiveness in robotics~\cite{Breazeal2003,Fan2013,Hoffman2009,lu2011position,Pinhanez97}. However, few works have been presented using theatre as environment to build and test sociability in robots. Some have used theatre to study timing in human-human interaction~\cite{Knight2011,Knight2011b}, others have focused on humanoid platforms that could be used as actors~\cite{Fan2009,Fan2013}, but none of them have worked on how to use theatre to improve emotion expression in totally autonomous robots.

With the context define above in mind, This work attack two potential issues in robotics via theatre. The fisrt, is the emotional expresiveness of the robot-actor taking only a few steps toward the notion of social robot-actor. The second, is the coordination process between a group of robot-actors. These potentialities are faced from the popular deliberative process in agent's theory called BDI(Belief, Desires and Intentions) which is in turn inspired by a respected theory of rational actions in humans. 

This papers is organized as follows. In section~\ref{sec:relatedwork} is reviewed the use of robots in theatre and section~\ref{sec:theatre} gives the highlights to be selected theatre as a test bed. Section~\ref{sec:architecture} describes the generalities of the architecture, and sections\ref{sec:description},~\ref{sec:belief},~\ref{sec:action}, and~\ref{sec:motivation} gives a more detail about each sub-system that conforms the architecture. Conclusions and further work are presented in section~\ref{sec:conclusions}. 

